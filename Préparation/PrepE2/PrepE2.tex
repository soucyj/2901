\documentclass{Preparation}
\def\titre{Préparation à l'examen 2}
\def\soustitre{À l'intention des étudiants du cours MAT-2901 : Mathématiques et technologie}
\def\auteur{J\'er\^ome Soucy}

\def\DateHeure{}
\def\Ponderation{}

\usepackage{commandesJS}
\usepackage{array}					% Pour mon tableau de la question 6.
\usepackage{multirow}				% Pour mon tableau de la question 6.
\usepackage{graphicx}
\usepackage{siunitx}
\sisetup{
	output-decimal-marker={,},
	group-separator={\,},
}
\begin{document}
\pagecouverture
\section{Détails de l'évaluation}
\begin{itemize}
	\item Date et heure : Sur le site du cours, rendez-vous dans la section \textit{Évaluations et résultats}~$\rightarrow$~\textit{Examen 2}
	\item Pondération : Sur le site du cours, rendez-vous dans la section \textit{Évaluations et résultats}~$\rightarrow$~\textit{Examen 2}
	\item Local : Sur le site du cours, rendez-vous dans la section \textit{Évaluations et résultats}~$\rightarrow$~\textit{Examen 2}
	\item Matière à l'étude
		\begin{itemize}
			\item Démonstration du théorème du point fixe de Banach
			\item Série 6 : Numération
			\item Série 7 : Cryptologie (1/3)
			\item Série 8 : Cryptologie (2/3)
			\item Série 9 : Cryptologie (3/3)
		\end{itemize}
    \item Matériel requis
		\begin{itemize}
			\item Crayon et efface
			\item Règle graduée en centimètres
            \item Calculatrice autorisée par la \textsc{fsg} (voir le site de cours pour connaître les modèles autorisés)
		\end{itemize}
	\item Aucun matériel ne sera fourni avec l'évaluation
\end{itemize}
\newpage
\section{Questions préparatoires}
\begin{question}
Montrez que si une contraction sur un espace métrique complet possède un point fixe, alors celui-ci est unique.
\end{question}
\begin{question}
   Donnez la taille, en kilooctets d'une image de résolution $1024\times 768$ dont le mode de description des couleurs utilise 16 teintes de gris.
\end{question}
\begin{question}
  Calculez l'inverse de la matrice
  \[\left(
  			\begin{tabular}{cc}
  				$4$ & $1$\\
  				$-1$ & $2$
  			\end{tabular}
  			\right).
  \]
  \end{question}
  \begin{question}
   Trouvez la valeur de $d$ où $d:=\textrm{pgcd}\,(3\,072, 96)$.
   \end{question}
    
  \begin{question}
  Trouvez $x,y\in\Z$ tels que $3\,072x+96y=\textrm{pgcd}\,(3\,072, 96)$.
  \end{question}
    
  \begin{question}
   Déterminez l'inverse multiplicatif de 7 modulo 41.
   \end{question}
    
  \begin{question}
  Trouvez la factorisation première de $32\,472$.
  \end{question}
  \begin{question}Trouvez le plus petit entier $x$ supérieur ou égal à 1000 satisfaisant l'équation $$4x\equiv 1~(\text{mod}~17).$$
  \end{question}
  \begin{question}
  Résolvez le système de congruence ci-dessous.
  \begin{eqnarray*}
      x\equiv & 1~(\text{mod}~8)\\
      3x-2\equiv & 0~(\text{mod}~7)\
  \end{eqnarray*}
   \end{question}
   \begin{question}
  Le savant irakien Al-Kindi, qui a vécu au neuvième siècle, s'est intéressé à la cryptologie. Plus précisément, il s'est intéressé au chiffrement par substitution monoalphabétique. Notons $E$ l'ensemble des lettres de l'alphabet\,$\left(E=\{a,b,c,\ldots,z\}\right)$. Une \textbf{permutation} de $E$ est une bijection de $E$ vers $E$. La matrice $2\times 26$ ci-dessous illustre une permutation de $E$. Par exemple, la lettre $a$ est envoyée sur la lettre $g$, la lettre $b$ sur la lettre $f$, etc.
    \begin{center}
    {\small
    \begin{displaymath}\left(
    \begin{tabular}{llllllllllllllllllll}
    $a$ & $b$ & $c$ & $d $&$ e $& $f$ & $g$ & $h$ & $i$ & $j$ & $k$ & $l$ & $m$\\
    $g$ & $f$ & $e $& $d$ & $c$ & $b $&$ a $& $h$ &$ i$ &$ j$ &$ z $&$ x $& $y $ 
    \end{tabular}
    \begin{tabular}{llllllllllllllllllll}
    $n$ & $o$ & $p$ & $q$ & $r$ & $s$ & $t$ & $u$ & $v$ & $w$ & $x$ & $y$ & $z $\\
    $k$ & $l$ & $m$ & $q$ & $r$ & $s$ & $t$ & $u$ & $v$ & $w $&$ p$ &$ o$ & $n$  
    \end{tabular}
    \right)
    \end{displaymath}}
    \end{center}
    Dans le cours, nous avons étudié certaines méthodes de cryptologie par substitution monoalphabétique, notamment le chiffre affine. Étant donnée une lettre de l'alphabet, on lui associe d'abord un nombre $x$ entre 0 et 25. On détermine ensuite un nombre naturel $\alpha\in\{1,2,\ldots,25\}$, vérifiant $\text{pgcd}\,(\alpha,26)=1$, et un nombre naturel $\beta\in\{0,1,2,\ldots,25\}$. On calcule le nombre $y:=\alpha x+\beta \mod 26$; celui compris entre 0 et 25. Le nombre $y$ est alors envoyé vers la lettre de l'alphabet qui lui correspond.
    
    \begin{parts}
     \part Combien y-a-t-il de permutations de $E$?
     \part Étant donné une transformation affine $\alpha x+\beta$, existe-t-il une permutation de $E$ qui lui est équivalente? Justifiez.
     \part Étant donné une permutation de $E$, existe-t-il une transformation affine $\alpha x+\beta$ qui lui est équivalente? Justifiez.
     \part Étant donné une transformation affine $\alpha x+\beta$, existe-il des lettres de l'alphabet qui ne seront pas modifiée par la transformation? Justifiez.
     \end{parts}
    \end{question}
\end{document}
